\section{Durchführung}
\label{sec:Durchführung}
Der Versuch besteht im wesentlichen aus zwei Teilen. Zwischen jeder neuen Messung muss der XY-Schreiber für die jeweiligen Messungen kalibriert werden.
Hierzu werden die x- und y-Minima so verschoben, dass diese in der unteren linken Bildecke liegen. Auch die MAxima werden so kalibriert, damit eine möglichst große und folglich genau Abbildung dargestellt werden kann.

\subsubsection{Bestimmung der Streuung der Elektronenenergie}
\label{sec:Bestimmung der Streuung der Elektronenenergie}
Der erste Teil des Versuches besteht darin die Streuung der Elektronen zu bestimmen.
Hierzu wird die Beschleunigungsspannung $U_{\text{B}}$ auf einen konstanten Wert gestellt und den X-Eingang des Schreibers die Bremsspannung $U_{\text{A}}$.
Jetzt wird die Bremsspannung erhöht und der XY-Schreiber zeichnet einen Graphen, welcher den vom Picometer gemessenen Strom gegen die Beschleunigungsspannung aufträgt.

\subsubsection{Franck-Hertz-Kurve}
\label{sec:Franck-Hertz-Kurve}
Der zweite Teil besteht darin den Stromverlauf und damit die Anzahl der Ladungen je nach Beschleunigungsspannung graphisch aufzutragen.
An den X-Eingang des Schreibers wird folglich $U_{\text{B}}$ angeschlossen. Die Temperatur $T$ wird auf zwischen $160-200\, \unit{\celsius}$ gestellt und es wird eine konstante Spannung $U_{\text{A}}$ angeschlossen.
Im Anschluss muss $U_{\text{B}}$ gleichmässig erhöht werden, sodass der Schreiber einen Graphen zeichnet. Dieser Vorgang wird bei gleicher Temperatur mit einer anderen Bremsspannung wiederholt.
Nachdem die beiden Graphen gezeichnet wurden, wird die Temperatur erhöht und es werden zwei weitere Graphen gezeichnet.


