\section{Ziel}
\label{sec:Ziel}

Mittels des Franck-Hertz-Versuchs gelang es 1914 eine Anregungsenergie des Quecksilber-Atoms zu bestimmen sowie einen Zusammenhang zwischen 
dieser und der Wellenlänge des emittierten Lichts herzustellen. Außerdem konnten die Bohrschen Postulate zum Aufbau der Elektronenhülle teilweise bestätigt 
werden. 

In diesem Versuch soll die Energieverteilung der verwendeten Elektronen und die Ionisierungsenergie von Quecksilber bestimmt werden. Zusätzlich 
dazu werden mehrere Franck-Hertz-Kurven bestimmt, um die Anregungsenergie des Hg-Atoms zu bestimmen