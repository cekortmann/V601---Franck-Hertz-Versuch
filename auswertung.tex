\section{Auswertung}
\label{sec:Auswertung}

\subsection{Fehlerrechnung}
\label{sec:Fehlerrechnung}
Für die Fehlerrechnung werden folgende Formeln aus der Vorlesung verwendet.
für den Mittelwert gilt
\begin{equation}
    \overline{x}=\frac{1}{N}\sum_{i=1}^N x_i ß\; \;\text{mit der Anzahl N und den Messwerten x} 
    \label{eqn:Mittelwert}
\end{equation}
Der Fehler für den Mittelwert lässt sich gemäß
\begin{equation}
    \increment \overline{x}=\frac{1}{\sqrt{N}}\sqrt{\frac{1}{N-1}\sum_{i=1}^N(x_i-\overline{x})^2}
    \label{eqn:FehlerMittelwert}
\end{equation}
berechnen.
Wenn im weiteren Verlauf der Berechnung mit der fehlerhaften Größe gerechnet wird, kann der Fehler der folgenden Größe
mittels Gaußscher Fehlerfortpflanzung berechnet werden. Die Formel hierfür ist
\begin{equation}
    \increment f= \sqrt{\sum_{i=1}^N\left(\frac{\partial f}{\partial x_i}\right)^2\cdot(\increment x_i)^2}.
    \label{eqn:GaussMittelwert}
\end{equation}


Für die verwendeten Temperaturen ergeben sich mittels \autoref{eqn:druck} der Sättigungsdampfdruck und mittels \autoref{eqn:weglaenge} die mittlere 
freie Weglänge, die beide in \autoref{tab: temp} dargestellt sind.
\begin{table}
    \centering
    \caption{Sättigungsdampfdruck und mittlere freie Weglänge.}
\begin{tabular}{c c c c}
    \toprule
        $T\mathbin{/}\unit{\celsius}$ &$T\mathbin{/}\unit{\kelvin}$ & $p_{\text{sät}}\mathbin{/}\symup{mbar}$ & $\bar{w} \mathbin{/} \unit{\cm}$ \\
    \midrule
    24.2 & 297.35 & 0.0050 & 0.5818\\
    140.2 & 413.35 & 3.2806 & 0.0009\\
    160.9 & 434.05 & 7.2525 & 0.0004\\
    162.1 & 435.25 & 7.5763 & 0.0004\\
    174.8 & 447.95 & 11.8570 & 0.0002 \\
    176.2 & 449.35 & 12.4378 & 0.0002 \\
     \bottomrule
    \end{tabular}
    \label{tab: temp}
\end{table}

Wenn die mittlere freie Weglänge nun mit dem Abstand $a=1\,\unit{\cm}$ zwischen Kathode und Beschleunigerelektrode verglichen wird, stellt man fest, dass die
mittlere freie Weglänge bei den verwendeten Temperaturen weitaus kleiner als dieser ist (Faktor 1000 bis 5000). Bei Zimmertemperatur von $24.2\,\unit{\celsius}$ ist die Differenz
zwischen a und $\bar{w}$ jedoch kleiner, was dazu führt die Stoßwahrscheinlichkeit sehr gering ist.

\subsection{Differentielle Energieverteilung}
Hier werden die ersten beiden Graphen des Anhangs betrachtet, wobei der erste Graph der Messung bei Zimmertemperatur entspricht. Um die Steigung der 
Graphen zu bestimmen, muss zuerst gezählt werden wie viele Skalenteile/Millimeter einem Volt auf der Skala $U_{\symup{A}}$ entsprechen. Dazu werden 
die Abstände zwischen den markierten Skalenpunkten gezählt und im Anschluss gemittelt. Diese Berechnung ist in \autoref{tab:skala1} dargestellt.
\begin{table}
    \centering
    \caption{Einteilung der Skala.}
\begin{tabular}{c c c}
    \toprule
         & Messreihe bei $T=297.35\,\unit{\kelvin}$ & Messreihe bei $T=413.35\,\unit{\kelvin}$ \\
    \midrule
    (0-1)V & 26 Skt & 31 Skt \\
    (1-2)V & 27 Skt & 25 Skt \\
    (2-3)V & 22 Skt & 25 Skt \\
    (3-4)V & 25 Skt & 25 Skt \\
    (4-5)V & 25 Skt & 25 Skt\\
    (5-6)V & 23 Skt &  23 Skt \\
    (6-7)V & 27 Skt & 23 Skt \\
    (7-8)V & 26 Skt & 25 Skt \\
    (8-9)V & 24 Skt & 25 Skt \\
    (9-10)V & 25 Skt & 19 Skt \\
    \midrule
    Mittelwert & 25.0 Skt/V & 24.6 Skt/V \\
     \bottomrule
    \end{tabular}
    \label{tab: temp}
\end{table}

Nun wird für jedes Intervall von 10 Skt die mittlere Steigung abgelesen, indem Steigungsdreiecke eingezeichnet werden. Diese befinden sich bereits auf der 
Abbildung im Anhang. Die Steigungen beider Messreihen befinden sich in \autoref{tab:steigung}.

\begin{table}
    \centering
    \caption{Bestimmung der Steigung der Graphen beider Messreihen.}
\begin{tabular}{c c | c c}
    \toprule
    & Steigung bei $T=297.35\,\unit{\kelvin}$ & & Steigung bei  $T=413.35\,\unit{\kelvin}$ \\
        Position / V & $\frac{\symup{\Delta}y}{\symup{\Delta}x} \mathbin{/} Skt_y/Skt_x$ &Position / V & $\frac{\symup{\Delta}y}{\symup{\Delta}x} \mathbin{/} Skt_y/Skt_x$ \\
    \midrule
    0.16 & 1.9 & 0.2 & 3.0 \\
    0.56 & 1.5 & 0.6 & 1.9 \\
    0.96 & 1.5 & 1.0 & 2.1 \\
    1.36 & 1.4 & 1.4 & 2.0 \\
    1.76 & 1.3 & 1.8 & 1.9 \\
    2.16 & 1.1 & 2.2 & 1.8 \\ 
    2.56 & 1.1 & 2.6 & 1.6 \\
    2.96 & 0.9 & 3.0 & 1.6 \\
    3.36 & 0.9 & 3.4 & 1.4 \\
    3.76 & 0.8 & 3.8 & 1.2 \\
    4.16 & 0.7 & 4.0 & 1.0 \\
    4.56 & 0.7 & &\\
    4.96 & 0.6 & &\\
    5.36 & 0.5 & &\\
    5.76 & 0.4 & &\\
    6.16 & 0.4 & &\\
    6.56 & 0.4 & &\\
    6.96 & 0.3 & &\\
    7.36 & 0.4 & &\\
    7.76 & 0.2 & &\\
     \bottomrule
    \end{tabular}
    \label{tab:steigung}
\end{table}

Die Steigung ist dabei proportional zu der Anzahl der Elektronen, während die Spannung $U_{\symup{A}}$ proportional zu der Energie ist. Für die Messreihe 
bei $T=297.35\,\unit{\kelvin}$ befindet sich die graphische Darstellung der Steigung in Abhängigkeit von der Bremsspannung in \autoref{fig:steigung20}.

\begin{figure}
    \centering
    \includegraphics[width = 12cm]{build/steigung20.pdf}
    \caption{Differentielle Energieverteilung bei $T=297.35\,\unit{\kelvin}$.}
    \label{fig:steigung20}
\end{figure}

Aus dem Graphen in \autoref{fig:steigung20} ergibt sich dabei ein sichtbares Kontaktpotential bei $U_{\symup{A}} = 6.96 \,\unit{\volt}$. Wenn zwischen
$U_{\symup{B}} = 10\,\unit{\volt}$ und diesem Wert eine Differenz gebildet wird, ergibt sich für das Kontaktpotential $K = 3.04\,\unit{\volt}$.

In \autoref{fig:steigung140} wird die Steigung in Abhängigkeit der Spannung $U_{\symup{A}}$ bei $T=413.35\,\unit{\kelvin}$ dargestellt.
\begin{figure}
    \centering
    \includegraphics[width = 12cm]{build/steigung140.pdf}
    \caption{Differentielle Energieverteilung bei $T=413.35\,\unit{\kelvin}$.}
    \label{fig:steigung140}
\end{figure}
Wenn die zweite Messung mit der vorherigen verglichen wird, fällt auf, dass deutlich mehr Elektronen mit Quecksilberatomen stoßen. Um eine qualitative
Aussage, auch bezüglich der Anregungsenergie von Quecksilber, treffen zu können, hätten mehr Messwerte aufgenommen werden müssen, sodass die 
Bremsspannung $U_{\symup{A}}$ auch über die Anregungsenergie von $E = 4.9 \, \symup{eV}$ \cite{anregung} hinausgeht.

\subsection{Franck-Hertz-Kurve} 
Die Franck-Hertz-Kurven bei $U_{\symup{A}} = 1\,\unit{\volt}$ und $U_{\symup{A}} = 2\,\unit{\volt}$ für zwei verschiedene Temperaturen sind ebenfalls 
im Anhang dargestellt.
Es werden erneut für jede Kurve die Skalenteile gezählt un dim Anschluss gemittelt. Für die Kurve bei $U_{\symup{A}} = 1\,\unit{\volt}$ und 
$T = 434.05 \,\unit{\kelvin}$ ergibt sich damit für den Mittelwert $(7.03 \pm 0.08) \symup{Skt/V}$. 
Genauso können die Abstände zwischen den einzelnen Maxima der Franck-Hertz-Kurve abgelesen und gemittelt werden. Diese Daten befinden sich in 
\autoref{tab:fh1max}.
\begin{table}
    \centering
    \caption{Abstände zwischen den Maxima der ersten Franck-Hertz-Kurve.}
\begin{tabular}{c c c}
    \toprule
        Maximum & Abstand in Skt & Abstand $U_1$ in V \\
    \midrule
    1 & 35 \pm 1 & 4.98 \pm 0.06 \\
    2 & 37 \pm 1 & 5.26 \pm 0.06 \\
    3 & 37 \pm 1 & 5.26 \pm 0.06 \\
    4 & 38 \pm 1 & 5.41 \pm 0.06 \\
    \midrule
    Mittelwert & & 5.23 \pm 0.04 \\
     \bottomrule
    \end{tabular}
    \label{tab:fh1max}
\end{table}

Die erste Anregungsenergie des Hg-Atoms ist nun der Mittelwert des Abstands zwischen aufeinanderfolgende Maxima multipliziert mit der Elementarladung
(siehe \autoref{eqn:U1}) 
\begin{equation*}
    (E_1-E_0) = (5.23 \pm 0.04) \,\unit{\eV} \; .
\end{equation*}
Mit \autoref{eqn:nu} und $\lambda = \frac{c}{\nu}$ ergibt sich für die Wellenlänge
\begin{equation*}
    \lambda =  (237.1 \pm 1.8)\,\unit{\nano\meter} \; .
\end{equation*}

Für die weiteren Franck-Hertz-Kurven wird das Verfahren genauso wiederholt.

\subsubsection{Franck-Hertz-Kurve bei $U_{\symup{A}} = 2\,\unit{\volt}$ und $T = 435.25 \,\unit{\kelvin}$}
Der Mittelwert der Skalenteile ergibt sich damit zu $(6.94 \pm 0.08) \symup{Skt/V}$. Die Abstände zwischen den Maxima werden in \autoref{tab:fh2max}
dargestellt.
\begin{table}
    \centering
    \caption{Abstände zwischen den Maxima der zweiten Franck-Hertz-Kurve.}
\begin{tabular}{c c c}
    \toprule
        Maximum & Abstand in Skt & Abstand $U_1$ in V \\
    \midrule
    1 & 37 \pm 1 & 5.04 \pm 0.06 \\
    2 & 38 \pm 1 & 5.33 \pm 0.06 \\
    3 & 42 \pm 1 & 5.48 \pm 0.06 \\
    \midrule
    Mittelwert & & 5.62 \pm 0.06 \\
     \bottomrule
    \end{tabular}
    \label{tab:fh2max}
\end{table}
Die erste Anregungsenergie ergibt sich zu 
\begin{equation*}
    (E_1-E_0) = (5.62 \pm 0.06) \,\unit{\eV} 
\end{equation*}
und die Wellenlänge des emittierten Lichts zu 
\begin{equation*}
    \lambda =  (220.6 \pm 2.4)\,\unit{\nano\meter} \; .
\end{equation*}